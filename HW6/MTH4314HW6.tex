% !TEX root = MTH4314HW6.tex
\documentclass[12pt]{article}
\usepackage[margin=1in]{geometry} 
\usepackage{amsmath,amsthm,amssymb,scrextend}
\usepackage{fancyhdr}
\pagestyle{fancy}

\newcommand{\cont}{\subseteq}
\usepackage{tikz}
\usepackage{pgfplots}
\usepackage{amsmath}
\usepackage[mathscr]{euscript}
\let\euscr\mathscr \let\mathscr\relax% just so we can load this and rsfs
\usepackage[scr]{rsfso}
\usepackage{amsthm}
\usepackage{amssymb}
\usepackage{enumitem}
\usepackage{multicol}
\usepackage{tcolorbox}
\usepackage{mdframed}
\usepackage{changepage}
\usepackage{soul}
\usepackage[colorlinks=true, pdfstartview=FitV, linkcolor=blue,
citecolor=blue, urlcolor=blue]{hyperref}

\DeclareMathOperator{\arcsec}{arcsec}
\DeclareMathOperator{\arccot}{arccot}
\DeclareMathOperator{\arccsc}{arccsc}
\newcommand{\Z}{\mathbb{Z}}
\newcommand{\R}{\mathbb{R}}
\newcommand{\C}{\mathbb{C}}
\newcommand{\F}{\mathbb{F}}
\newcommand{\Q}{\mathbb{Q}}
\newcommand{\N}{\mathbb{N}}
\newcommand{\ddx}{\frac{d}{dx}}
\newcommand{\dfdx}{\frac{df}{dx}}
\newcommand{\ddxp}[1]{\frac{d}{dx}\left( #1 \right)}
\newcommand{\dydx}{\frac{dy}{dx}}
\let\ds\displaystyle
\newcommand{\intx}[1]{\int #1 \, dx}
\newcommand{\intt}[1]{\int #1 \, dt}
\newcommand{\defint}[3]{\int_{#1}^{#2} #3 \, dx}
\newcommand{\imp}{\Rightarrow}
\newcommand{\un}{\cup}
\newcommand{\inter}{\cap}
\newcommand{\ps}{\mathscr{P}}
\newcommand{\set}[1]{\left\{ #1 \right\}}
\newtheorem*{sol}{Solution}
\newtheorem*{claim}{Claim}
\newtheorem*{prop}{Proposition}
\newtheorem{problem}{Problem}
\numberwithin{problem}{section} % Reset problem counter in each section
\theoremstyle{remark}  % Style for remarks
\newtheorem*{remark}{Remark}
\newenvironment{answer}
    {\begin{adjustwidth}{0pt}{0pt}}
    {\end{adjustwidth}}

\begin{document}
 
% EVERYTHING ABOVE THIS LINE IS JUST PREABLE, NO NEED TO MESS WITH IT.__________________________________________________________________________________________
%
\lhead{Rawson Duplantis}
\chead{MTH 4314: Abstract Algebra}
\rhead{\today}

% Rest of Section 3.2: Exercise 63 + Sections 4.2 & 4.3
\stepcounter{section}
\section{Groups}
\setcounter{subsection}{2}
\subsection{Subgroups and Direct Products}
\stepcounter{subsubsection}
\subsubsection{Direct Products}
% 3.2: 63
\setcounter{problem}{62}
%\begin{problemgroup}
    \begin{problem}[Subgroups of $\Z$ and $\Z_n$]
        Complete the following:
        \begin{enumerate}[label=(\alph*)]
            \item If $H \neq \{0\}$ is a subgroup of $\Z$, let $k$ be the minimal element of $\N \cap H$. Show $k$ exists and that $H = \langle k \rangle$. Hint: If $m\in H$, use the Euclidean Algorithm to write $m=kq+r$ with $0\leq r < k$ and show $kq \in H$ so that $r\in H$.
            \item Conclude that the set of subgroups of $\Z$ is $\{\langle k \rangle = k\Z \mid k \in \Z_{\geq 0}\}$.
            \item For $n\in\N$, show that every subset of $\Z_n$ is of the form $\langle [k] \rangle$ for $0\leq k < n$.
            \item Show $\langle [k] \rangle = \langle [(k,n)] \rangle$. Hint: For $\subseteq$, use $(k,n) \mid k$. For $\supseteq$, write $(k,n) = kx+ny$.
            \item Conclude that the set of subgroups of $\mathbb{Z}_n$ is $\{\langle [k] \rangle \mid k \in \mathbb{N}\text{ and } k\mid n\}$.
            \item Find all subgroups of $\mathbb{Z}_{15}$.      
        \end{enumerate}
    \end{problem}
    \begin{answer}
        For part (a), we'll first show that $\langle k \rangle \subseteq H$ and then show that $H \subseteq \langle k \rangle$. For $\langle k \rangle \subseteq H$, we know that $k\in H$. Because $H\leqslant \Z$, we can say $H$ contains all integer multiples of $k$: thus $\langle k \rangle \subseteq H$. Next, for $H \subseteq \langle k \rangle$, we write any element $m\in H$ as $kq+r$. We then can say $m,kq\in H \Rightarrow r \in H$. However, because $k$ is defined as the minimal non-zero element, $r$ must be $0$; therefore $H \subseteq \langle k \rangle$. Finally, $H \subseteq \langle k \rangle$ and $\langle k \rangle \subseteq H$ imply $\langle k \rangle = H$. \vspace{5pt} \\ For part (b), we know from part (a) that any subgroup $H$ is equal to the subgroup generated by the minimal positive element $k$. Because this minimal element can be any $k\in\N$, the comprehensive set of subgroups is simply the set of $\langle k \rangle \ \forall k\in\Z_{> 0}$. This definition also includes the trivial case as $\langle 0 \rangle$ is built by $0\Z$, which is simply $\{0\}$. Therefore the comprehensive set of subgroups is the set of $\langle k \rangle \ \forall k\in\Z_{\geq0}$. \vspace{5pt} \\ For part (c), we know that any subgroup of $\Z_n$ is a subgroup operating on one of the equivalence classes represented by 
    \end{answer}
%\end{problemgroup}

\subsection{Morphisms}
\subsubsection{Definitions and Examples}
% 4.2: 92bdeg, 93abc, 94acdeg, 95
\setcounter{problem}{91}
%\begin{problemgroup}
    \begin{problem}
        Show the following maps are homomorphisms:
        \begin{enumerate}[label=(\alph*)]
            \stepcounter{enumi}
            \item For $m\in\Z$ and $n\in\N$, $\varphi \operatorname{:} \Z_n \to \Z_n$ given by $\varphi([k])=m[k]$.
            \stepcounter{enumi}
            \item For $m\in\Z$ and $n\in\N$, $\varphi\operatorname{:}U_n \to U_n$ given by $\varphi([k])=[k]^m$.
            \item $\varphi\operatorname{:}\R\to\R^+$ given by $\varphi(x)=e^x$.
            \stepcounter{enumi}
            \item $\theta\operatorname{:}GL(n,\F)\to GL(n,\F)$ given by $\theta(g)=(g^{-1})^T$.
        \end{enumerate}
    \end{problem}
    \begin{answer}
        Answer here..
    \end{answer}
%\end{problemgroup}
%\begin{problemgroup}
    \begin{problem}
        Show the following maps are not homomorphisms:
        \begin{enumerate}[label=(\alph*)]
            \item $\varphi\operatorname{:}\Z\to\Z$ given by $\varphi(k)=k+1$.
            \item $\varphi\operatorname{:}\R\to\R$ given by $\varphi(x)=x^2$.
            \item $\varphi\operatorname{:}\R\to\R^\times$ given by $\varphi(x)=2x$.
        \end{enumerate}
    \end{problem}
    \begin{answer}
        Answer here...
    \end{answer}
%\end{problemgroup}
%\begin{problemgroup}
    \begin{problem}
        Show the following groups are isomorphic.
        \begin{multicols}{2}
            \begin{enumerate}[label=(\alph*)]
                \item $\Z\cong 2\Z$.
                \stepcounter{enumi}
                \item For $n\in\N$, $\{z\in\C \mid z^n=1\}\cong \Z_n$.
                \item $\R\cong \R^+$. Hint: $e^x$.
                \item $U_7\cong\Z_2\times\Z_3$.
                \stepcounter{enumi}
                \item $\R\cong U(2,\R)=\{\begin{pmatrix}
                    1 & x \\
                    0 & 1
                \end{pmatrix} \mid x,y\in\R^\times\}$.
            \end{enumerate}
        \end{multicols}
    \end{problem}
    \begin{answer}
        Answer here...
    \end{answer}
%\end{problemgroup}
%\begin{problemgroup}
    \begin{problem}
        Show the following groups are not isomorphic.
        \begin{enumerate}[label=(\alph*)]
            \item $\Z_4 \not\cong \Z_5$.
            \item $S_3 \not\cong \Z_6$.
            \item $\Z_4 \not\cong \Z_2\times\Z_2$.
            \item $R^\times \not\cong \R.$ Hint: Count the solutions to $x^2=1$ in $\R^\times$ and to $2x=0$ in $\R$.
            \item $\Z \not\cong \Q$.
        \end{enumerate}
    \end{problem}
    \begin{answer}
        Answer here...
    \end{answer}
%\end{problemgroup}

\subsubsection{Basic Properties}
% 4.3 96, 99abcd, 104
%\begin{problemgroup}
    \begin{problem}
        Calculate the kernels and images of the following homomorphisms.
        \begin{enumerate}[label=(\alph*)]
            \item $\varphi\operatorname{:} \Z_{10}\to\Z_{10}$ by $\varphi([k]) = 2[k]$.
            \item $\varphi\operatorname{:} U_{10}\to U_{10}$ by $\varphi([k])=[k]^2$.
            \item $\varphi\operatorname{:} D_n \to \{\pm 1\}$ by $\varphi(R_j)=1$ and $\varphi(W_j)=-1$.
        \end{enumerate}
    \end{problem}
    \begin{answer}
        Answer here...
    \end{answer}
%\end{problemgroup}
\setcounter{problem}{98}
%\begin{problemgroup}
    \begin{problem}[Homomorphisms from $\Z$]
        Complete the following:
        \begin{enumerate}[label=(\alph*)]
            \item If $G$ is a group and $\varphi\operatorname{:} \Z \to G$ is a homomorphism, show $\varphi(k)=\varphi(1)^k$ for $k\in\Z$.
            \item If $\varphi'\operatorname{:}\Z\to G$ is another homomorphism, show $\varphi = \varphi'$ if and only if $\varphi(1)=\varphi'(1)$.
            \item If $g\in G$ show the map $\varphi\operatorname{:}\Z\to G$ given by $\varphi(k)=g^k$ is a homomorphism.
            \item Conclude that the set of homomorphisms from $\Z$ to $G$ is in bijection with the set of elements of $G$.
        \end{enumerate}
    \end{problem}
    \begin{answer}
        Answer here...
    \end{answer}
%\end{problemgroup}
\setcounter{problem}{103}
%\begin{problemgroup}
    \begin{problem}
        Suppose $G$ is a group with $S\subseteq G$ and $G=\langle S \rangle$. If $\varphi, \varphi \operatorname{:} G \to H$ are homomorphisms satisfying $\varphi(s)=\varphi'(s)$ for all $s\in S$, show $\varphi=\varphi'$.
    \end{problem}
    \begin{answer}
        Answer here...
    \end{answer}
%\end{problemgroup}

\end{document}