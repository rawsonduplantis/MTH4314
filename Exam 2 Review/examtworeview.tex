% !TEX root = examtworeview.tex
\documentclass[12pt]{article}
\usepackage[margin=1in]{geometry} 
\usepackage{amsmath,amsthm,amssymb,scrextend}
\usepackage{fancyhdr}
\pagestyle{fancy}

\newcommand{\cont}{\subseteq}
\usepackage{tikz}
\usepackage{pgfplots}
\usepackage{amsmath}
\usepackage[mathscr]{euscript}
\let\euscr\mathscr \let\mathscr\relax% just so we can load this and rsfs
\usepackage[scr]{rsfso}
\usepackage{amsthm}
\usepackage{amssymb}
\usepackage{enumitem}
\usepackage{multicol}
\usepackage{tcolorbox}
\usepackage{mdframed}
\usepackage{changepage}
\usepackage{soul}
\usepackage[colorlinks=true, pdfstartview=FitV, linkcolor=blue,
citecolor=blue, urlcolor=blue]{hyperref}

\DeclareMathOperator{\arcsec}{arcsec}
\DeclareMathOperator{\arccot}{arccot}
\DeclareMathOperator{\arccsc}{arccsc}
\newcommand{\Z}{\mathbb{Z}}
\newcommand{\R}{\mathbb{R}}
\newcommand{\C}{\mathbb{C}}
\newcommand{\F}{\mathbb{F}}
\newcommand{\Q}{\mathbb{Q}}
\newcommand{\N}{\mathbb{N}}
\newcommand{\ddx}{\frac{d}{dx}}
\newcommand{\dfdx}{\frac{df}{dx}}
\newcommand{\ddxp}[1]{\frac{d}{dx}\left( #1 \right)}
\newcommand{\dydx}{\frac{dy}{dx}}
\let\ds\displaystyle
\newcommand{\intx}[1]{\int #1 \, dx}
\newcommand{\intt}[1]{\int #1 \, dt}
\newcommand{\defint}[3]{\int_{#1}^{#2} #3 \, dx}
\newcommand{\imp}{\Rightarrow}
\newcommand{\un}{\cup}
\newcommand{\inter}{\cap}
\newcommand{\ps}{\mathscr{P}}
\newcommand{\set}[1]{\left\{ #1 \right\}}
\newtheorem*{sol}{Solution}
\newtheorem*{claim}{Claim}
\newtheorem*{prop}{Proposition}
\newtheorem{problem}{Problem}
\newtheorem{defn}{Definition}
\newtheorem{thm}{Theorem}
\newtheorem{exer}{Exercise}
\newtheorem{lma}{Lemma}
\newtheorem{crlly}{Corollary}
\numberwithin{problem}{section} % Reset problem counter in each section
\numberwithin{defn}{section} % Reset problem counter in each section
\numberwithin{thm}{section} % Reset problem counter in each section
\numberwithin{exer}{section} % Reset problem counter in each section
\numberwithin{lma}{section} % Reset problem counter in each section
\numberwithin{crlly}{section} % Reset problem counter in each section
\theoremstyle{remark}  % Style for remarks
\newtheorem*{remark}{Remark}
\newenvironment{answer}
    {\begin{adjustwidth}{0pt}{0pt}}
    {\end{adjustwidth}}

\begin{document}
 
% EVERYTHING ABOVE THIS LINE IS JUST PREABLE, NO NEED TO MESS WITH IT.__________________________________________________________________________________________
%
\lhead{Rawson Duplantis}
\chead{MTH 4314: Abstract Algebra}
\rhead{\today}

The following is a compiled review for the second exam according to the study guide.

\renewcommand{\thesubsection}{\Alph{subsection}}

\setcounter{section}{1}  % Start section numbering at 2
\section{Groups}
\setcounter{subsection}{0}  % (Optional) Reset subsection counter
\subsection{Definitions}

\setcounter{exer}{71}
\begin{exer}[Normalizer $N(S)$]
    Let $G$ be a group and let $S\subseteq G$ be nonempty. The \ul{normalizer} of $S$ in $G$ is $N(S)=N_G(S)=\{g\in G \mid gSg^{-1}=S\}$\dots
\end{exer}

\begin{exer}[Commutator]
    Suppose $G$ is a group and let $G'$ be the subgroup generated by $\{g_1g_2g_1^{-1}g_2^{-1} \mid g_i \in G\}$, called the \ul{commutator subgroup} of $G$\dots
\end{exer}

\begin{defn}[Torsion]
    Let $G$ be an abelian group and let $T$ be the set of all elements of $G$ with finite order; $T$ is a subgroup called the \ul{torsion subgroup}. 
\end{defn}

\setcounter{defn}{19}
\begin{defn}[Homomorphism, Isomorphism, Automorphism]
    Let $G$ and $H$ be groups and let $\varphi\operatorname{:}G \to H$ be a function.
    \begin{enumerate}
        \item If $\varphi(g_1g_2)=\varphi(g_1)\varphi(g_2)$ for all $g_1,g_2\in G$, then $\varphi$ is a \ul{homomorphism}.
        \item A bijective (one-to-one and onto) homomorphism is called an \ul{isomorphism}. In that case, $G$ amd $H$ are said to be isomorphic and we write $G \cong H$.
        \item If $G=H$, an isomorphism is also called an \ul{automorphism}.
    \end{enumerate}
\end{defn}

\setcounter{defn}{21}
\begin{defn}[Conjugation Map]
    Let $G$ be a group and let $g,h\in G$. The \ul{conjugation map} $c_g\operatorname{:} G\to G$ is defined by $c_g(h) = ghg^{-1}$.
\end{defn}

\setcounter{defn}{23}
\begin{defn}[Image and Kernel]
    Let $\varphi\operatorname{:}G\to H$ be a homomorphism between groups.
    \begin{enumerate}
        \item The \ul{image} of $\varphi$ is $\operatorname{Im}\varphi = \{\varphi(g) \mid g \in G\}$.
        \item The \ul{kernel} of $\varphi$ is $\operatorname{ker}\varphi = \{g\in G \mid \varphi(g)=e_H\}$.
    \end{enumerate}
\end{defn}

\setcounter{defn}{26}
\begin{defn}[Cosets]
    Let $G$ be a group and let $H$ be a subgroup of $G$. Let $g\in G$.
    \begin{enumerate}
        \item The \ul{left coset} of $g$ with respect to $H$ is $gH=\{gh \mid h\in H\}$.
        \item If $C$ is a left coset with respect to $H$ and $C=gH$, then $g$ is called a \ul{representative} of $C$.
        \item The set of left cosets is denoted $G/H$ and is called the \ul{quotient} of $G$ by $H$.
        \item The \ul{index} of $H$ in $G$ is $|G/H|$ and is denoted $[G\operatorname{:}H]$.
    \end{enumerate}
\end{defn}

\setcounter{defn}{29}
\begin{defn}[Normality]
    Let $G$ be a group, let $H$ be a subgroup of $G$, and let $g\in G$.
    \begin{enumerate}
        \item The \ul{right coset} of $g$ with respect to $H$ is $Hg= \{hg \mid h \in H\}$.
        \item $H$ is called \ul{normal} if $gHg^{-1}\subseteq H$ for all $g\in G$ where $gHg^{-1}=\{ghg^{-1}\mid h\in H\}$.
    \end{enumerate}
\end{defn}

\setcounter{thm}{30}
\begin{thm}[Normality T.F.A.E.s]
    The following are equivalent:
    \begin{enumerate}
        \item $H$ is normal.
        \item $gHg^{-1}=H$ for all $g\in G$.
        \item Left cosets are right cosets; i.e., $gH=Hg$ for all $g\in G$.
        \item $gH \subseteq Hg$ for each $g \in G$.
        \item $Hg \subseteq gH$ for each $g \in G$.
        \item The equation $(g_1H)(g_2H)=(g_1g_2)H$ gives a well-defined binary operation on $G/H$ where $g_1,g_2\in G$.
        \item $ghg^{-1} \in H$ for all $g\in G, h\in H$.
    \end{enumerate}
\end{thm}

\setcounter{defn}{31}
\begin{defn}
    Let $G$ be a group and let $H$ be a normal subgroup of $G$. Under the bilinear operation $(g_1H)(g_2H)=(g_1g_2)H$ for $g_1,g_2\in G$, $G/H$ is a group and is called the \ul{quotient group} or \ul{factor group}.
\end{defn}

\setcounter{exer}{150}
\begin{exer}
    Let $N$ and $H$ be groups along with a homomorphism $\varphi\operatorname{:} H\to \operatorname{Aut}(N)$. For $n\in N$ and $h\in H$, write $\varphi_h(n)$ for $(\varphi(h))(n)$. Define the \ul{semidirect product} of $N$ and $H$ as $$N \rtimes H = \{(n,h) \mid n\in N, h\in H\}$$ with a group law given by $$(n_1,h_1)(n_2,h_2)=(n_1\varphi_{h_1}(n_2),h_1h_2)\dots$$
\end{exer}

\subsection{Book Proofs}
\setcounter{thm}{22}
\begin{thm}
    Let $G$ be a group and let $g \in G$. Then $c_g$ is an automorphism.
\end{thm}
\begin{proof}
    We must show that the conjugation map $c_g$ is first, a homomorphism, second, an isomorphism, and third, from $G$ to $G$.
    \begin{enumerate}
        \item \ul{Homomorphism}: Let $h_1,h_2\in G$. We know that $$c_g(h_1h_2)=gh_1h_2g^{-1}=gh_1g^{-1}gh_2g^{-1}=c_g(h_1)c_g(h_2)$$ therefore $c_g$ is a homomorphism.
        \item \ul{Isomorphism}: Because $c_{g^{-1}}$ is a valid inverse of $c_g$, we know that $c_g$ is a bijection and thus also an isomorphism.
        \item \ul{$G \to G$}: As $g,h_1,h_2\in G$, $c_g$ thus maps $G\to G$ and is consequently an automorphism.
    \end{enumerate}
\end{proof}

\stepcounter{thm}
\begin{thm}
    Let $\varphi\operatorname{:} G \to H$ be a homomorphism between groups and let $g\in G$.
    \begin{enumerate}
        \setcounter{enumi}{2}
        \item $\varphi(e_G)=e_h$.
        \item $\varphi(g^{-1})=\varphi(g)^{-1}$.
        \item $\varphi$ is one-to-one if and only if $\ker \varphi=\{e_G\}$.
    \end{enumerate}
\end{thm}
\begin{proof}
    For part (3), because $\varphi$ is a homomorphism, we can say \begin{align*}
        \varphi(e_G) &= \varphi(e_G)\varphi(e_G) \\
        \varphi(e_G)\varphi(e_G)^{-1} &= \varphi(e_G)\varphi(e_G)\varphi(e_G)^{-1} \\
        e_He_H &= \varphi(e_G) \\
        e_H &= \varphi(e_G).
    \end{align*}
    For part (4), we can similarly say
    \begin{align*}
        e_H = \varphi(e_G) = \varphi(gg^{-1}) &= \varphi(g)\varphi(g^{-1}) \\
        &\Rightarrow \varphi(g)^{-1} = \varphi(g^{-1}).
    \end{align*}
    For part (8), we must show that $\varphi$ is one-to-one implies that $\ker \varphi=\{e_G\}$, and vice versa.
    \begin{enumerate}
        \item \ul{$\varphi$ is one-to-one $\Rightarrow \ker \varphi = \{e_G\}$}: If we suppose that $\varphi$ is one-to-one, we can say that given an element $g\in G$ such that $\varphi(g)=e_H\in H$, we know that $g=e_G$ as $e_G\mapsto e_H$ in any one-to-one mapping.
        \item \ul{$\ker \varphi = \{e_G\}\Rightarrow \varphi$ is one-to-one}: Suppose we have $g_1,g_2\in G$ such that $\varphi(g_1)=\varphi(g_2)$. We can then say that $\varphi(g_1)\varphi(g_2^{-1})=\varphi(g_2g_2^{-1})\Rightarrow \varphi(g_1g_2^{-1}) = e_H$ which implies $g_1=g_2$, which forces $\varphi$ to be one-to-one.
    \end{enumerate}
    Thus "$\varphi$ is one-to-one" $\Leftrightarrow \ker \varphi = \{e_G\}$.
\end{proof}

\setcounter{lma}{26}
\begin{lma}
    Let $G$ be a group, let $H \leq G$, and let $g_1,g_2\in G$. Then $g_1H=g_2H$ if an only if $g_2=g_1h$ for some $h\in H$.
\end{lma}
\begin{proof}
    To prove their biconditional relationship, we must prove that $g_1H=g_2H\Rightarrow g_2=g_1h$ for some $h\in H$ and $g_2=g_1h$ for some $h\in H \Rightarrow g_1H=g_2H$.
    \begin{enumerate}
        \item \ul{$g_1H=g_2H \Rightarrow \exists h\in H,\ g_2=g_1h$}: If we suppose $g_1H=g_2H$, we know that $e\in H$ implies $g_2\in g_2H$; consequently $g_2\in g_1H$ therefore $\exists h\in H,\ g_2=g_1h$.
        \item \ul{$\exists h\in H,\ g_2=g_1h\Rightarrow g_1H=g_2H$}: If we then suppose that $\exists h\in H,\ g_2=g_1h$, we know that $g_2h=g_1hh$ implies $g_2H \subseteq g_1H$ as $H$ is closed. For the reverse, we can rewrite the starting expression as $g_2h^{-1}=g_1$ which can be similarly manipulated as $g_2h^{-2}=g_1h^{-1}$ which implies $g_2H \supseteq g_1 H$ when all $h$ are taken into account. Therefore these two subsets are equal, $g_1H=g_2H$.
    \end{enumerate}
    Thus $g_1H=g_2H \Leftrightarrow \exists h\in H,\ g_2=g_1h$.
\end{proof}

\setcounter{crlly}{28}
\begin{crlly}
    Let $G$ be a finite group.
    \begin{enumerate}
        \item If $g\in G$, then $|g|$ divides $|G|$.
        \item In particular, $g^{|G|}=e$.
    \end{enumerate}
\end{crlly}
\begin{proof}
    For part (1), we know that $|g|=|\langle g \rangle|$ and that $\langle g \rangle$ is a subgroup of $G$; therefore $|g|$ does divides $G$ by Lagrange's theorem as $G$ is a finite group. For part (2), we can say that $|G|=|g|k$ for some $k\in \Z_{>0}$. Therefore $g^{|G|}=g^{|g|k}=(g^{|g|})^k=e^k=e$.
\end{proof}

\setcounter{thm}{34}
\begin{thm}[First Isomorphism Theorem]
    Let $\varphi\operatorname{:}G\to H$ be a homomorphism of groups.
    \begin{enumerate}
        \item $\ker\varphi$ is a normal subgroup of $G$.
    \end{enumerate}
\end{thm}
\begin{proof}
    For part (1), suppose $k\in \ker \varphi$ and $g\in G$. By definition of the kernel: $$\varphi(gkg^{-1})=\varphi(g)\varphi(k)\varphi(g)^{-1}=\varphi(g)e_H\varphi(g)^{-1}=e_H.$$ Therefore $g\ker\varphi g^{-1}=\ker\varphi$ and $\ker\varphi$ is normal.   
\end{proof}

\setcounter{lma}{38}
\begin{lma}[Cauchy's Theorem for Finite Abelian Groups]
    Let $G$ be a finite abelian group and let $p$ be a positive prime with $p\mid |G|$. Then $G$ has an element of order $p$.
\end{lma}
\begin{proof}
    We can express $G$ as $\{g_1,\dots,g_n\}$ with each elements order, $|g_i|$, represented by $d_i$. We can define a subgroup $H$ as a direct product of $n$ number of $\Z_{d_i}$, expressed as $\prod_{i=1}^{n}\Z_{d_i}$. We can then define a map $\varphi\operatorname{:}H\to G$ by $\varphi(k_1,\dots,k_n)=g^{k_1}\dots g^{k_n}$; this is a well-defined, surjective homomorphism as $G$ is abelian. Therefore, by the First Isomorphism Theorem, $|H|=|\ker\varphi||G|$. If $p$ divides $|G|$, $p$ must also then divide $|H|$ and consequently $\prod^{n}_{i=1}d_i$. Therefore $p$ must divide at least one $d_i$ and $g=g_i^{d_i/p}$ which is only true if $d_i=p$; thus $p$ is the order of at least one element $g_i$.
\end{proof}

\subsection{Homework Exercises}

\setcounter{exer}{52}
\begin{exer}
    Show the following subsets are groups:
    \begin{enumerate}[label=(\alph*)]
        \setcounter{enumi}{1}
        \item $\{5^a\mid a \in \mathbb{Q}\} \subseteq \mathbb{R}^\times$.
        \item $k\mathbb{Z}_n=\{k[m]\mid[m]\in\mathbb{Z}_n\}\subseteq \mathbb{Z}_n$ where $k\in \mathbb{Z}$ and $n\in\mathbb{N}$.
        \setcounter{enumi}{6}
        \item $SL(n,\mathbb{R})\subseteq GL(n,\mathbb{R})$.       
    \end{enumerate}
\end{exer}
\begin{answer}
    In order to demonstrate the given subsets are groups, we need to show that their operation is closed, that all elements are associative, that there exists an identity, and that each element has an inverse. 
    \begin{enumerate}[label=(\alph*)]
        \stepcounter{enumi}
        \item The subset from part (b) is a group as it:
        \begin{enumerate}[label=(\roman*)]
            \item is closed, as $(5^a)\cdot(5^n) = 5^{a+n}$, $a+n\in\mathbb{Q}\ \forall n\in \mathbb{Q}$;
            \item is associative, as $(5^{a}5^{b})5^{c}=5^{a+b+c}=5^{a}(5^{b}5^{c})\ \forall b,c\in\mathbb{Q}$;
            \item contains an identity, $5^0$, as $5^a5^0=5^{a+0}=5^{a}$, and;
            \item has an inverse for every element, as $5^a5^{-a}=5^{a-a}=5^0$.
        \end{enumerate}
        \item The subset from part (c) is a group as it:
        \begin{enumerate}[label=(\roman*)]
            \item is closed, as $k[i]+k[j]=k[i+j]\in k\mathbb{Z}_n$;
            \item is associative, as $k[i]+(k[j]+k[l]) = (k[i]+k[j])+k[l] = k[i + j + l]$;
            \item contains an identity, as $k[0]\equiv 0 \pmod{n}$, and;
            \item has an inverse for every element, as $k[i]+k[-i]=k[0] \equiv 0 \pmod{n}$.
        \end{enumerate}
        \setcounter{enumi}{6}
        \item The subset from part (g) is a group as it:
        \begin{enumerate}[label=(\roman*)]
            \item is closed, as two $n\times n$ invertible matrices with $\det = 1$ will produce an invertible matrix with $\det = 1$;
            \item is associative, as all matrices in the general linear group are associative;
            \item contains an identity, the identity matrix with $\det(I_n) = 1$, and;
            \item has an inverse for every element, as $\det(A) = 1 \Rightarrow \det(A^{-1}) = \frac{1}{\det(A)} = 1^{-1} = 1$.
        \end{enumerate}
    \end{enumerate}
\end{answer}

\setcounter{exer}{56}
\begin{exer}
    Evaluate the following:
    \begin{enumerate}[label=(\alph*)]
        \item $([3]_7,[5]_6) \cdot ([2]_7, [5]_6) \in U_7\times U_6$.
        \item $([2]_4, [3]_5, [6]_7) + ([3]_4, [2]_5, [1]_7) \in \mathbb{Z}_4 \times \mathbb{Z}_5 \times \mathbb{Z}_7$.
    \end{enumerate}
\end{exer}
\begin{answer}
    For part (a), we perform component-wise multiplication: $$[3]_7\cdot[2]_7=[6]_7,\ [5]_6\cdot[5]_6=[1]_6.$$ Therefore the answer in $U_7\times U_6$ is $([6]_7,[1]_6)$. For part (b) we perform component-wise addition: $$[2]_4+[3]_4=[1]_4,\ [3]_5+[2]_5=[0]_5,\ [6]_7+[1]_7=[0]_7.$$ Therefore the answer in $\Z_4 \times \Z_5 \times \Z_7$ is $([1]_4,[0]_5,[0]_7)$.
\end{answer}

\setcounter{exer}{62}
\begin{exer}[Subgroups of $\Z$ and $\Z_n$]
    Complete the following:
    \begin{enumerate}[label=(\alph*)]
        \setcounter{enumi}{5}
        \item Find all subgroups of $\mathbb{Z}_{15}$.      
    \end{enumerate}
\end{exer}
\begin{answer}
    The set of subgroups of $\Z_n$ is $\{\langle [k] \rangle \mid k\in \N \text{ and } k \mid n\}$. Therefore the set of all subgroups of $\Z_{15}$ is $\{\langle [0] \rangle, \langle [1] \rangle, \langle [3] \rangle, \langle [5] \rangle\}$.
\end{answer}

\setcounter{exer}{95}
\begin{exer}
    Show the following groups are not isomorphic.
    \begin{enumerate}[label=(\alph*)]
        \item $\Z_4 \not \cong \Z_5$.
        \item $S_3 \not \cong \Z_6$.
        \item $\Z_4 \not \cong \Z_2 \times \Z_2$.
        \item $\R^\times \not \cong \R$.
        \item $\Z \not \cong \Q$.
    \end{enumerate}
\end{exer}

\subsection{Homework Proofs}

\subsection{New Exercises}

\end{document}