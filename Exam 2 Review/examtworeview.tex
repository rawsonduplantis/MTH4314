% !TEX root = examtworeview.tex
\documentclass[12pt]{article}
\usepackage[margin=1in]{geometry} 
\usepackage{amsmath,amsthm,amssymb,scrextend}
\usepackage{fancyhdr}
\pagestyle{fancy}

\newcommand{\cont}{\subseteq}
\usepackage{tikz}
\usepackage{pgfplots}
\usepackage{amsmath}
\usepackage[mathscr]{euscript}
\let\euscr\mathscr \let\mathscr\relax% just so we can load this and rsfs
\usepackage[scr]{rsfso}
\usepackage{amsthm}
\usepackage{amssymb}
\usepackage{enumitem}
\usepackage{multicol}
\usepackage{tcolorbox}
\usepackage{mdframed}
\usepackage{changepage}
\usepackage{soul}
\usepackage[colorlinks=true, pdfstartview=FitV, linkcolor=blue,
citecolor=blue, urlcolor=blue]{hyperref}

\DeclareMathOperator{\arcsec}{arcsec}
\DeclareMathOperator{\arccot}{arccot}
\DeclareMathOperator{\arccsc}{arccsc}
\newcommand{\Z}{\mathbb{Z}}
\newcommand{\R}{\mathbb{R}}
\newcommand{\C}{\mathbb{C}}
\newcommand{\F}{\mathbb{F}}
\newcommand{\Q}{\mathbb{Q}}
\newcommand{\N}{\mathbb{N}}
\newcommand{\ddx}{\frac{d}{dx}}
\newcommand{\dfdx}{\frac{df}{dx}}
\newcommand{\ddxp}[1]{\frac{d}{dx}\left( #1 \right)}
\newcommand{\dydx}{\frac{dy}{dx}}
\let\ds\displaystyle
\newcommand{\intx}[1]{\int #1 \, dx}
\newcommand{\intt}[1]{\int #1 \, dt}
\newcommand{\defint}[3]{\int_{#1}^{#2} #3 \, dx}
\newcommand{\imp}{\Rightarrow}
\newcommand{\un}{\cup}
\newcommand{\inter}{\cap}
\newcommand{\ps}{\mathscr{P}}
\newcommand{\set}[1]{\left\{ #1 \right\}}
\newtheorem*{sol}{Solution}
\newtheorem*{claim}{Claim}
\newtheorem*{prop}{Proposition}
\newtheorem{problem}{Problem}
\newtheorem{defn}{Definition}
\newtheorem{thm}{Theorem}
\newtheorem{exer}{Exercise}
\numberwithin{problem}{section} % Reset problem counter in each section
\numberwithin{defn}{section} % Reset problem counter in each section
\numberwithin{thm}{section} % Reset problem counter in each section
\numberwithin{exer}{section} % Reset problem counter in each section
\theoremstyle{remark}  % Style for remarks
\newtheorem*{remark}{Remark}
\newenvironment{answer}
    {\begin{adjustwidth}{0pt}{0pt}}
    {\end{adjustwidth}}

\begin{document}
 
% EVERYTHING ABOVE THIS LINE IS JUST PREABLE, NO NEED TO MESS WITH IT.__________________________________________________________________________________________
%
\lhead{Rawson Duplantis}
\chead{MTH 4314: Abstract Algebra}
\rhead{\today}

The following is a compiled review for the second exam according to the study guide.

\renewcommand{\thesubsection}{\Alph{subsection}}

\setcounter{section}{1}  % Start section numbering at 2
\section{Groups}
\setcounter{subsection}{0}  % (Optional) Reset subsection counter
\subsection{Definitions}

\setcounter{exer}{71}
\begin{exer}[Normalizer $N(S)$]
    Let $G$ be a group and let $S\subseteq G$ be nonempty. The \ul{normalizer} of $S$ in $G$ is $N(S)=N_G(S)=\{g\in G \mid gSg^{-1}=S\}$\dots
\end{exer}

\begin{exer}[Commutator]
    Suppose $G$ is a group and let $G'$ be the subgroup generated by $\{g_1g_2g_1^{-1}g_2^{-1} \mid g_i \in G\}$, called the \ul{commutator subgroup} of $G$\dots
\end{exer}

\begin{defn}[Torsion]
    Let $G$ be an abelian group and let $T$ be the set of all elements of $G$ with finite order; $T$ is a subgroup called the \ul{torsion subgroup}. 
\end{defn}

\setcounter{defn}{19}
\begin{defn}[Homomorphism, Isomorphism, Automorphism]
    Let $G$ and $H$ be groups and let $\varphi\operatorname{:}G \to H$ be a function.
    \begin{enumerate}
        \item If $\varphi(g_1g_2)=\varphi(g_1)\varphi(g_2)$ for all $g_1,g_2\in G$, then $\varphi$ is a \ul{homomorphism}.
        \item A bijective (one-to-one and onto) homomorphism is called an \ul{isomorphism}. In that case, $G$ amd $H$ are said to be isomorphic and we write $G \cong H$.
        \item If $G=H$, an isomorphism is also called an \ul{automorphism}.
    \end{enumerate}
\end{defn}

\setcounter{defn}{21}
\begin{defn}[Conjugation Map]
    Let $G$ be a group and let $g,h\in G$. The \ul{conjugation map} $c_g\operatorname{:} G\to G$ is defined by $c_g(h) = ghg^{-1}$.
\end{defn}

\setcounter{defn}{23}
\begin{defn}[Image and Kernel]
    Let $\varphi\operatorname{:}G\to H$ be a homomorphism between groups.
    \begin{enumerate}
        \item The \ul{image} of $\varphi$ is $\operatorname{Im}\varphi = \{\varphi(g) \mid g \in G\}$.
        \item The \ul{kernel} of $\varphi$ is $\operatorname{ker}\varphi = \{g\in G \mid \varphi(g)=e_H\}$.
    \end{enumerate}
\end{defn}

\setcounter{defn}{26}
\begin{defn}[Cosets]
    Let $G$ be a group and let $H$ be a subgroup of $G$. Let $g\in G$.
    \begin{enumerate}
        \item The \ul{left coset} of $g$ with respect to $H$ is $gH=\{gh \mid h\in H\}$.
        \item If $C$ is a left coset with respect to $H$ and $C=gH$, then $g$ is called a \ul{representative} of $C$.
        \item The set of left cosets is denoted $G/H$ and is called the \ul{quotient} of $G$ by $H$.
        \item The \ul{index} of $H$ in $G$ is $|G/H|$ and is denoted $[G\operatorname{:}H]$.
    \end{enumerate}
\end{defn}

\setcounter{defn}{29}
\begin{defn}[Normality]
    Let $G$ be a group, let $H$ be a subgroup of $G$, and let $g\in G$.
    \begin{enumerate}
        \item The \ul{right coset} of $g$ with respect to $H$ is $Hg= \{hg \mid h \in H\}$.
        \item $H$ is called \ul{normal} if $gHg^{-1}\subseteq H$ for all $g\in G$ where $gHg^{-1}=\{ghg^{-1}\mid h\in H\}$.
    \end{enumerate}
\end{defn}

\setcounter{thm}{30}
\begin{thm}[Normality T.F.A.E.s]
    The following are equivalent:
    \begin{enumerate}
        \item $H$ is normal.
        \item $gHg^{-1}=H$ for all $g\in G$.
        \item Left cosets are right cosets; i.e., $gH=Hg$ for all $g\in G$.
        \item $gH \subseteq Hg$ for each $g \in G$.
        \item $Hg \subseteq gH$ for each $g \in G$.
        \item The equation $(g_1H)(g_2H)=(g_1g_2)H$ gives a well-defined binary operation on $G/H$ where $g_1,g_2\in G$.
        \item $ghg^{-1} \in H$ for all $g\in G, h\in H$.
    \end{enumerate}
\end{thm}

\setcounter{defn}{31}
\begin{defn}
    Let $G$ be a group and let $H$ be a normal subgroup of $G$. Under the bilinear operation $(g_1H)(g_2H)=(g_1g_2)H$ for $g_1,g_2\in G$, $G/H$ is a group and is called the \ul{quotient group} or \ul{factor group}.
\end{defn}

\setcounter{exer}{150}
\begin{exer}
    Let $N$ and $H$ be groups along with a homomorphism $\varphi\operatorname{:} H\to \operatorname{Aut}(N)$. For $n\in N$ and $h\in H$, write $\varphi_h(n)$ for $(\varphi(h))(n)$. Define the \ul{semidirect product} of $N$ and $H$ as $$N \rtimes H = \{(n,h) \mid n\in N, h\in H\}$$ with a group law given by $$(n_1,h_1)(n_2,h_2)=(n_1\varphi_{h_1}(n_2),h_1h_2)\dots$$
\end{exer}

\subsection{Book Proofs}

\subsection{Homework Exercises}

\subsection{Homework Proofs}

\subsection{New Exercises}

\end{document}