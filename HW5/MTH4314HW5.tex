% !TEX root = MTH4314HW5.tex
\documentclass[12pt]{article}
\usepackage[margin=1in]{geometry} 
\usepackage{amsmath,amsthm,amssymb,scrextend}
\usepackage{fancyhdr}
\pagestyle{fancy}

\newcommand{\cont}{\subseteq}
\usepackage{tikz}
\usepackage{pgfplots}
\usepackage{amsmath}
\usepackage[mathscr]{euscript}
\let\euscr\mathscr \let\mathscr\relax% just so we can load this and rsfs
\usepackage[scr]{rsfso}
\usepackage{amsthm}
\usepackage{amssymb}
\usepackage{enumitem}
\usepackage{multicol}
\usepackage{tcolorbox}
\usepackage{mdframed}
\usepackage{changepage}
\usepackage{soul}
\usepackage[colorlinks=true, pdfstartview=FitV, linkcolor=blue,
citecolor=blue, urlcolor=blue]{hyperref}

\DeclareMathOperator{\arcsec}{arcsec}
\DeclareMathOperator{\arccot}{arccot}
\DeclareMathOperator{\arccsc}{arccsc}
\newcommand{\ddx}{\frac{d}{dx}}
\newcommand{\dfdx}{\frac{df}{dx}}
\newcommand{\ddxp}[1]{\frac{d}{dx}\left( #1 \right)}
\newcommand{\dydx}{\frac{dy}{dx}}
\let\ds\displaystyle
\newcommand{\intx}[1]{\int #1 \, dx}
\newcommand{\intt}[1]{\int #1 \, dt}
\newcommand{\defint}[3]{\int_{#1}^{#2} #3 \, dx}
\newcommand{\imp}{\Rightarrow}
\newcommand{\un}{\cup}
\newcommand{\inter}{\cap}
\newcommand{\ps}{\mathscr{P}}
\newcommand{\set}[1]{\left\{ #1 \right\}}
\newtheorem*{sol}{Solution}
\newtheorem*{claim}{Claim}
\newtheorem*{prop}{Proposition}
\newtheorem{problem}{Problem}
\numberwithin{problem}{section} % Reset problem counter in each section
\theoremstyle{remark}  % Style for remarks
\newtheorem*{remark}{Remark}
\newenvironment{answer}
    {\begin{adjustwidth}{0pt}{0pt}}
    {\end{adjustwidth}}

\begin{document}
 
% EVERYTHING ABOVE THIS LINE IS JUST PREABLE, NO NEED TO MESS WITH IT.__________________________________________________________________________________________
%
\lhead{Rawson Duplantis}
\chead{MTH 4314: Abstract Algebra}
\rhead{\today}

% Chapter 2 -- 2: 38abefg, 39, 40ab, 45; 3.1 53bcgh, 54ae (for e read v_0 in R, v_0 neq 0), 55b, 56ab, 63 (for a replace Euclidean with Division and for c replace subset with subgroup); 3.2: 57

\setcounter{section}{1}
\section{Groups}
\setcounter{subsection}{1}
\subsection{Basic Properties and Order}
\setcounter{problem}{37}
%\begin{problemgroup}
\renewcommand{\thefootnote}{\fnsymbol{footnote}}
    \begin{problem}
        For each element $g$ of the listed groups below, find the order of $g$, $|g|$.
        \begin{multicols}{3}
            \begin{enumerate}[label=(\alph*)]
                \item $[3] \in (\mathbb{Z}_{15}, +)$.
                \item $[3] \in (U_{10}, \cdot)$.
                \setcounter{enumi}{4}
                \item $\begin{pmatrix}
                    1 & 2 & 3 & 4 & 5 \\
                    2 & 1 & 3 & 5 & 4
                \end{pmatrix} \in S_5$.
                \item $R_2\in D_3$
                \item $\begin{pmatrix}
                    [1] & [1] \\
                    [0] & [1]
                \end{pmatrix} \in GL(2,\mathbb{Z}_2)$
            \end{enumerate}
        \end{multicols}
    \end{problem}
    \begin{answer}
        For part (a), $|[3]|=5$ as $3 + 3 + 3 + 3 + 3 \equiv 0 \pmod{15}$. For part (b), $|[3]|=4$ as $3^4=81\equiv 1 \pmod{10}$. For part (e), the order of the given $\sigma\in S_5$ is 2 as $\sigma^2=e$: $$\begin{pmatrix}
            1 & 2 & 3 & 4 & 5 \\
            2 & 1 & 3 & 5 & 4
        \end{pmatrix} \operatorname{\circ} \begin{pmatrix}
            1 & 2 & 3 & 4 & 5 \\
            2 & 1 & 3 & 5 & 4
        \end{pmatrix} = \begin{pmatrix}
            1 & 2 & 3 & 4 & 5 \\
            2 & 1 & 3 & 5 & 4 \\
            1 & 2 & 3 & 4 & 5
        \end{pmatrix}\footnotemark= \text{Id}.
        $$ \footnotetext{This may be incorrect notation but it gets the idea across.} For part (f), $|R_2|=3$ as a 240\textdegree\ rotation of a triangle must be repeated three times in order for the triangle to reach its original orientation, $e$. For part (g): $$ \begin{pmatrix}
                [1] & [1] \\
                [0] & [1]
            \end{pmatrix}
            \begin{pmatrix}
                [1] & [1] \\
                [0] & [1]
            \end{pmatrix} = 
            \begin{pmatrix}
                [1] + [0] & [1] + [1] \\
                [0] + [0] & [0] + [1]
            \end{pmatrix} =
            \begin{pmatrix}
                [1] & [0] \\
                [0] & [1]
            \end{pmatrix};
        $$ therefore the order of the matrix is 2.
    \end{answer}
%\end{problemgroup}
%\begin{problemgroup}
    \begin{problem}
        In each infinite group below, find all elements of finite order:
        \begin{enumerate}[label=(\alph*)]
            \item ($\mathbb{R}$, +),
            \item ($\mathbb{R}^\times$, $\cdot$),
            \item ($\mathbb{C}^\times$, $\cdot$),
            \item $D(n,\mathbb{R}) = \{\text{diag}(c_1,\dots,c_n)\operatorname{|}c_i\in \mathbb{R}^\times, 1 \leq i \leq n\}$.
        \end{enumerate}
    \end{problem}
    \begin{answer}
        For part (a), the set of finite-order elements of $(\mathbb{R}, +)$ is $\{0\}$. For part (b), the set of finite-order elements of $(\mathbb{R^\times}, \cdot)$ is $\{1, -1\}$ as $(-1)^2=1$. For part (c), the set of finite-order elements contains all $n$-th roots of unity. For part (d), the set of finite-order elements of the group of diagonals composed of $\mathbb{R}$ is the trivial set composed of 1s along the diagonal of the $n\times n$ matrix.\footnote{In research, I found that finite-order elements are called \emph{torsion elements} and that groups can be classified as a \emph{torsion group} if it only contains torsion elements. Reportedly this term comes from algebraic topology, but the connection is so complex that I can't understand how they're related---something about twisting a space?}
    \end{answer}
%\end{problemgroup}
%\begin{problemgroup}
    \begin{problem}
        Let $G$ be a group and let $g\in G$.
        \begin{enumerate}[label=(\alph*)]
            \item Show $|g^{-1}|=|g|$.
            \item For $h\in G$, show $|hgh^{-1}|=|g|$.
            \item If $|g|< \infty$, show $g^{-1}=g^{|g|-1}$.
        \end{enumerate}
    \end{problem}
    \begin{answer}
        The following proofs will address each part respectively:
        \begin{proof}
            We know that by Theorem 2.11 part (2) that $g^{n_1}=g^{n_2}$ if and only if $n_1 \equiv n_2 \pmod{|g|}$. By the definition of an element's order, $g^n=e$ for some minimal positive exponent $n$. We can then say $g^n = e \Rightarrow (g^n)^{-1} = e^{-1} \Rightarrow g^{-n} = e$ which can only be true if $n$ and $-n$ are equivalent $\pmod{n}$. Therefore, because $-n\equiv n\equiv 0 \pmod{n}\text{, }|g^{-1}|=|g|$.
        \end{proof}
        \begin{proof}
            We can start by rewriting $|hgh^{-1}|=|g|$ as $(hgh^{-1})^n$ which can be simplified: $$hg(h^{-1}h)g(h^{-1}h)gh^{-1}\dots = hg^nh^{-1}.$$ We can also say that $hg^nh^{-1}=e$ is true only when $g^n=e$ as it allows $$hg^nh^{-1}=heh^{-1}=hh^{-1}=e.$$ To demonstrate they are both the minimal exponent, consider $(hgh^{-1})^m=e$. We can simplify it to $e$ as $h^{-1}(hgh^{-1})^mh=g^m=e$, thus $m=n$ and $|hgh^{-1}|=|g|$.
        \end{proof}
        \begin{proof}
            By definition, $g^{|g|}=e$. Multiplying both sides by $g^{-1}$, we get $g^{|g|}g^{-1}=eg^{-1} \Rightarrow g^{|g|-1}=g^{-1}$ by Theorem 2.9 part (1).
        \end{proof}
    \end{answer}
%\end{problemgroup}
\setcounter{problem}{44}
%\begin{problemgroup}
    \begin{problem}[$g^2=e\implies\text{Abelian}$]
        Suppose $G$ is a group so that $g^2=e$ for every $g\in G$. Show that $G$ is abelian. Hint: Show $g\in G$ implies $g^{-1}=g$ and then apply this fact to the product of two elements.
    \end{problem}
    \begin{answer}
        \begin{proof}
            Suppose $G$ is a group such that $g^2=e\ \forall g\in G$. Thus $g^2g^{-1}=eg^{-1}\Rightarrow g = g^{-1}$ by Theorem 2.9 part (1). Next, using another element $h$ and Theorem 2.9 part (2), we can say: $$
                (gh)^{-1} = h^{-1}g^{-1} = hg \text{ and } (gh)^{-1} = (gh)
            $$ as every element is its own inverse in this given $G$. Therefore $G$ must be abelian.
        \end{proof}
    \end{answer}
%\end{problemgroup}

% Chapter 2 -- 2: 38abefg, 39, 40ab, 45; 3.1 53bcgh, 54ae (for e read v_0 in R, v_0 neq 0), 55b, 56ab, 63 (for a replace Euclidean with Division and for c replace subset with subgroup); 3.2: 57

\subsection{Subgroups and Direct Products}
\subsubsection{Subgroups}
\setcounter{problem}{52}
%\begin{problemgroup}
    \begin{problem}
        Show the following subsets are groups:
        \begin{enumerate}[label=(\alph*)]
            \setcounter{enumi}{1}
            \item $\{5^a\mid a \in \mathbb{Q}\} \subseteq \mathbb{R}^\times$.
            \item $k\mathbb{Z}_n=\{k[m]\mid[m]\in\mathbb{Z}_n\}\subseteq \mathbb{Z}_n$ where $k\in \mathbb{Z}$ and $n\in\mathbb{N}$.
            \setcounter{enumi}{6}
            \item $SL(n,\mathbb{R})\subseteq GL(n,\mathbb{R})$.
            \item $\{T \in GL(n,\mathbb{R}) \mid Tv_0=\lambda v_0, \lambda \in \mathbb{R}^{+}\} \subseteq GL(n,\mathbb{R})$ for some fixed $v_0\in \mathbb{R}^n$.
        \end{enumerate}
    \end{problem}
    \begin{answer}
        In order to demonstrate the given subsets are groups, we need to show that their operation is closed, that all elements are associative, that there exists an identity, and that each element has an inverse. 
        \begin{enumerate}[label=(\alph*)]
            \stepcounter{enumi}
            \item The subset from part (b) is a group as it:\begin{enumerate}[label=(\roman*)]
                \item is closed, as $(5^a)\cdot(5^n) = 5^{a+n}$, $a+n\in\mathbb{Q}\ \forall n\in \mathbb{Q}$;
                \item is associative, as $(5^{a}5^{b})5^{c}=5^{a+b+c}=5^{a}(5^{b}5^{c})\ \forall b,c\in\mathbb{Q}$;
                \item contains an identity, $5^0$, as $5^a5^0=5^{a+0}=5^{a}$, and;
                \item has an inverse for every element, as $5^a5^{-a}=5^{a-a}=5^0$.
            \end{enumerate}
            \item The subset from part (c) is a group as it:
            \begin{enumerate}[label=(\roman*)]
                \item is closed, as $k[i]+k[j]=k[i+j]\in k\mathbb{Z}_n$;
                \item is associative, as $k[i]+(k[j]+k[l]) = (k[i]+k[j])+k[l] = k[i + j + l]$;
                \item contains an identity, as $k[0]\equiv 0 \pmod{n}$, and;
                \item has an inverse for every element, as $k[i]+k[-i]=k[0] \equiv 0 \pmod{n}$.
            \end{enumerate}
            \setcounter{enumi}{6}
            \item The subset from part (g) is a group as it:
            \begin{enumerate}[label=(\roman*)]
                \item is closed, as two $n\times n$ invertible matrices with $\det = 1$ will produce an invertible matrix with $\det = 1$;
                \item is associative, as all matrices in the general linear group are associative;
                \item contains an identity, the identity matrix with $\det(I_n) = 1$, and;
                \item has an inverse for every element, as $\det(A) = 1 \Rightarrow \det(A^{-1}) = \frac{1}{\det(A)} = 1^{-1} = 1$.
            \end{enumerate}
            \item The subset from part (h) is a group as it:
            \begin{enumerate}[label=(\roman*)]
                \item is closed, as $(ST)v_0 = S(Tv_0) = S(\lambda v_0) = \lambda Sv_0 = \lambda\mu v_0$;
                \item is associative, as all elements in the general linear group are associative;
                \item contains an identity, the identity matrix: $I_nv_0 = 1v_0$, and;
                \item has an inverse for every element, as $v_0=T^{-1}Tv_0=T^{-1}\lambda v_0 = \lambda T^{-1}v_0 \Rightarrow T^{-1}v_0=\lambda^{-1}v_0$.
            \end{enumerate}
        \end{enumerate}
    \end{answer}
%\end{problemgroup}
%\begin{problemgroup}
    \begin{problem}
        Show the following are not subgroups:
        \begin{enumerate}[label=(\alph*)]
            \item $\{5^a \mid a \in \mathbb{Q}^+\} \subseteq \mathbb{R}^\times$.
            \setcounter{enumi}{4}
            \item For $v_0\in \mathbb{R}^n$, $\{T \in GL(n,\mathbb{R}) \mid Tv_0=2v_0\} \subseteq GL(n,\mathbb{R})$.
        \end{enumerate}
    \end{problem}
    \begin{answer}
        The subsets are not groups if they fail to uphold one of the four axioms discussed above. The subset from part (a) fails to be a group as the exclusion of $-a$ from $Q^{+}$ removes all possible non-trivial inverses from the set. The subset from part (e) fails to be a group as it excludes the identity matrix, which never maps to an eigenvalue of $2$.
    \end{answer}
%\end{problemgroup}
%\begin{problemgroup}
    \begin{problem}
        Calculate the centralizers of the following subsets:
        \begin{enumerate}[label=(\alph*)]
            \stepcounter{enumi}
            \item $\{\begin{pmatrix}
                1 & 1 \\
                0 & 1
            \end{pmatrix}\} \subseteq GL(2,\mathbb{R})$.
        \end{enumerate}
    \end{problem}
    \begin{answer}
        Let A=\begin{pmatrix} 1 & 1 \\ 0 & 1 \end{pmatrix}. We wish to compute the centralizer of \(A\) in \(GL(2,\mathbb{R})\), namely
        \[
        C_{GL(2,\mathbb{R})}(A)=\{X\in GL(2,\mathbb{R}) \mid XA=AX\}.
        \]
        
        Take an arbitrary \(X=\begin{pmatrix} a & b \\ c & d \end{pmatrix}\in GL(2,\mathbb{R})\). Then
        \[
        XA=\begin{pmatrix} a & b \\ c & d \end{pmatrix}\begin{pmatrix} 1 & 1 \\ 0 & 1 \end{pmatrix}
        =\begin{pmatrix} a & a+b \\ c & c+d \end{pmatrix},
        \]
        and
        \[
        AX=\begin{pmatrix} 1 & 1 \\ 0 & 1 \end{pmatrix}\begin{pmatrix} a & b \\ c & d \end{pmatrix}
        =\begin{pmatrix} a+c & b+d \\ c & d \end{pmatrix}.
        \]
        
        For \(XA=AX\), we equate the corresponding entries:
        \begin{enumerate}[label=(\roman*)]
            \item From the \((1,1)\)-entry: \(a = a+c\) implies \(c=0\).
            \item From the \((1,2)\)-entry: \(a+b = b+d\) implies \(a = d\).
            \item The \((2,1)\)-entry yields \(c=c\), which is automatically satisfied.
            \item The \((2,2)\)-entry gives \(c+d = d\); with \(c=0\) this holds automatically.
        \end{enumerate}
        
        Thus, every matrix \(X\) commuting with \(A\) must be of the form
        \[
        X=\begin{pmatrix} a & b \\ 0 & a \end{pmatrix},
        \]
        where \(a\in \\mathbb{R}^\times\) (to ensure that \(X\) is invertible) and \(b\in \mathbb{R}\).
        
        Hence, the centralizer of \(A\) in \(GL(2,\mathbb{R})\) is
        \[
        C_{GL(2,\mathbb{R})}(A)=\left\{\begin{pmatrix} a & b \\ 0 & a \end{pmatrix} \mid a\in \mathbb{R}^\times,\; b\in \mathbb{R}\right\}.
        \]
        \end{answer}
        
%\end{problemgroup}
%\begin{problemgroup}
    \begin{problem}
        Show the following:
        \begin{enumerate}[label=(\alph*)]
            \item $\langle[2]\rangle=U_5$.
            \item $\langle 3,11 \rangle = \mathbb{Z}$.
        \end{enumerate}
    \end{problem}
    \begin{answer}
        Answer here...
    \end{answer}
%\end{problemgroup}
\setcounter{problem}{62}
%\begin{problemgroup}
    \begin{problem}[Subgroups of $\mathbb{Z}$ and $\mathbb{Z}_n$] 
        Complete the following:
        \begin{enumerate}[label=(\alph*)]
            \item If $H\neq\{0\}$ is a subgroup of $\mathbb{Z}$, let $k$ be the minimal element of $\mathbb{N}\cap H$. Show $k$ exists and that $H=\langle k\rangle$. Hint: If $m\in H$, use the \st{Euclidean} Division Algorithm to write $m=kq+r$ with $0\leq r < k$ and show $kq\in H$ so that $r \in H$.
            \item Conclude that the set of subgroups of $\mathbb{Z}$ is $\{\langle k\rangle=k\mathbb{Z} \mid k\in\mathbb{Z}_{\geq 0}\}$.
            \item For $n\in \mathbb{N}$, show that every \st{subset} subgroup of $\mathbb{Z}_n$ is of the form $\langle k \rangle$ for $0 \leq k < n$.
            \item Show $\langle [k] \rangle = \langle [(k,n)] \rangle$. Hint: For $\subseteq$, use $(k,n)\operatorname{\mid} k$. For $\supseteq$, write $(k,n)=kx+ny$.
            \item Conclude that the set of subgroups of $\mathbb{Z}_n$ is $\{\langle [k] \rangle \mid k \in \mathbb{N}\text{ and } k\mid n\}$.
            \item Find all subgroups of $\mathbb{Z}_{15}$.
        \end{enumerate}
    \end{problem}
    \begin{answer}
        Answer here..
    \end{answer}
%\end{problemgroup}

\subsubsection{Direct Products}
\setcounter{problem}{56}
%\begin{problemgroup}
    \begin{problem}
        Evaluate the following:
        \begin{enumerate}[label=(\alph*)]
            \item $([3]_7,[5]_6) \cdot ([2]_7, [5]_6) \in U_7\times U_6$.
            \item $([2]_4, [3]_5, [6]_7) + ([3]_4, [2]_5, [1]_7) \in \mathbb{Z}_4 \times \mathbb{Z}_5 \times \mathbb{Z}_7$.
        \end{enumerate}
    \end{problem}
    \begin{answer}
        Answer here...
    \end{answer}
%\end{problemgroup}

\end{document}