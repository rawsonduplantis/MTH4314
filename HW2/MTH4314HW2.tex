% !TEX root = MTH4314HW2.tex
\documentclass[12pt]{article}
\usepackage[margin=1in]{geometry} 
\usepackage{amsmath,amsthm,amssymb,scrextend}
\usepackage{fancyhdr}
\pagestyle{fancy}

\newcommand{\cont}{\subseteq}
\usepackage{tikz}
\usepackage{pgfplots}
\usepackage{amsmath}
\usepackage[mathscr]{euscript}
\let\euscr\mathscr \let\mathscr\relax% just so we can load this and rsfs
\usepackage[scr]{rsfso}
\usepackage{amsthm}
\usepackage{amssymb}
\usepackage{enumitem}
\usepackage{multicol}
\usepackage{tcolorbox}
\usepackage{mdframed}
\usepackage{changepage}
\usepackage[colorlinks=true, pdfstartview=FitV, linkcolor=blue,
citecolor=blue, urlcolor=blue]{hyperref}

\DeclareMathOperator{\arcsec}{arcsec}
\DeclareMathOperator{\arccot}{arccot}
\DeclareMathOperator{\arccsc}{arccsc}
\newcommand{\ddx}{\frac{d}{dx}}
\newcommand{\dfdx}{\frac{df}{dx}}
\newcommand{\ddxp}[1]{\frac{d}{dx}\left( #1 \right)}
\newcommand{\dydx}{\frac{dy}{dx}}
\let\ds\displaystyle
\newcommand{\intx}[1]{\int #1 \, dx}
\newcommand{\intt}[1]{\int #1 \, dt}
\newcommand{\defint}[3]{\int_{#1}^{#2} #3 \, dx}
\newcommand{\imp}{\Rightarrow}
\newcommand{\un}{\cup}
\newcommand{\inter}{\cap}
\newcommand{\ps}{\mathscr{P}}
\newcommand{\set}[1]{\left\{ #1 \right\}}
\newtheorem*{sol}{Solution}
\newtheorem*{claim}{Claim}
\newtheorem*{prop}{Proposition}
\newtheorem{problem}{Problem}
\numberwithin{problem}{section} % Reset problem counter in each section
\theoremstyle{remark}  % Style for remarks
\newtheorem*{remark}{Remark}
\newenvironment{answer}
    {\begin{adjustwidth}{0pt}{0pt}}
    {\end{adjustwidth}}

\begin{document}
 
% EVERYTHING ABOVE THIS LINE IS JUST PREABLE, NO NEED TO MESS WITH IT.__________________________________________________________________________________________
%
\lhead{Rawson Duplantis}
\chead{MTH 4314: Abstract Algebra}
\rhead{\today}
Note: unrequired parts of multi-part problems are listed obfuscated as $\dots$ to recognize they are multi-parted.

\section{Arithmetic}
\subsection{Integers}
\setcounter{subsubsection}{4}
\subsubsection{Fundamental Theorem of Arithmetic}

%\begin{problemgroup}
\setcounter{problem}{23}
    \begin{problem}
        Let $a,m,n\in\mathbb{Z}$ with $(m,n)=1$. Show $(a,mn)=(a,m)(a,n)$. Hint: Use Theorem 1.8 and recall that $m$ and $n$ have no common divisors.
    \end{problem}
    \begin{answer}
        \begin{proof}
            Suppose we have $a,m,n\in\mathbb{Z}$ such that $(m,n)=1$, $(a,m)$, and $(a,n)$. We can then express using Theorem 1.8 these greatest common divisors in their prime factorizations $$(a,m)=\prod_{i=0}^{M}p_i^{min\{a_i,m_i\}}\text{ and }(a,n)=\prod_{j=0}^{N}p_j^{min\{a_j,n_j\}}.$$ Since $(m,n)=1$, we know that their prime factorizations are unique from one another and that they each will contribute separate primes to $m\times n$ yielding $$(a,mn)=\prod_{i=0}^{M}p_i^{min\{a_i,m_i\}}\times\prod_{j=0}^{N}p_j^{min\{a_j,n_j\}}.$$ This cleanly becomes $(a,mn)=(a,m)(a,n)$ when the prime factorizations are re-expressed in $\gcd$ form.
        \end{proof}
    \end{answer}
%\end{problemgroup}
\vspace{5pt}
%\begin{problemgroup}
    \begin{problem}
        Let $p$ be a positive prime.
        \begin{enumerate}[label=(\alph*)]
            \item If $p\mid a^n$ for $a\in\mathbb{Z}$ and $n\in\mathbb{N}$, then $p^n\mid a^n$. Hint: Show $p\mid a$ first.
            \item Show there are no $a,b\in\mathbb{Z}^{\times}$ satisfying $a^2=pb^2$. Hint: If you could solve $a^2=pb^2$, show you may also assume $(a,b)=1$ by dividing. Then show $p\mid a$ and then that $p\mid b$ to get a contradiction. Alternative Hint: For (b) compare the exponents of $p$ on the LHS and the RHS of the equation $a^2=pb^2$.
            \item Show there is no $r\in\mathbb{Q}$ satisfying $r^2=p$; i.e., show $\sqrt{p}\notin\mathbb{Q}$.
        \end{enumerate}
    \end{problem}
    \begin{answer}
        The problem will be broken into three separate proofs for each subproblem.
        \begin{proof}
            Suppose $p\mid a^n$ for some $a\in\mathbb{Z}, n \in \mathbb{Z}_{>0}$ given positive prime $p$. We know that the sequence of $a$'s must contain a factor of $p$ according to our supposition, which is only possible if $p=a$ or $p\mid a$. Because $p=a\Rightarrow p\mid a$, we can say across all valid cases that $p\mid a^n \Rightarrow p\mid a$. From this we can say
            \begin{align*}
                p|a &\Rightarrow pk=a \exists k \in \mathbb{Z} \\
                &\Rightarrow (pk)^n = a^n \\
                &\Rightarrow p^nk^n = a^n \\
                &\Rightarrow p^nl = a^n,\ l=k^n\in \mathbb{Z}
            \end{align*}
            Therefore $p^n$ divides $a^n$.
        \end{proof}
        \begin{proof}
            Suppose, along with the original suppositions above, that $a^2=pb^2$ for some $a,b\in \mathbb{Z}^{\times}$. We can then quickly say that $\frac{a^2}{b^2}=p$. Starting with the first requirement of $p$, we know that $\frac{a^2}{b^2}$ must be an integer yielding three valid possibilities:
            \begin{enumerate}[label=(\roman*)]
                \item $a^2=b^2$ would always result in the integer one;
                \item $a^2>b^2,\ b=1$ will always result in the integer $a^2$, and;
                \item $a^2>b^2,\ b>1, (a,b)>1$, i.e. the fraction will only result in an integer when the numerator and the denominator share a common factor.
            \end{enumerate}
            All of these possibilities, however, do not result in a positive prime $p$. The first option is invalid as all primes are greater than 1. The second option is invalid as $\frac{a^2}{b^2}=(a,b)^2$ guarantees the fraction will yield a composite number (the square of their greatest common divisor). Finally, the third option is invalid for the same reason as two, albeit more obviously: $a^2$ can never be prime. Therefore $a^2\neq pb^2 \ \forall a,b\in\mathbb{Z}^\times$.
        \end{proof}
        \begin{proof}
            Suppose $\exists r\in \mathbb{Q}$ such that $r^2=p \Leftrightarrow \sqrt{p}\in \mathbb{Q}$. Because $r\in\mathbb{Q}$, we can say $\exists a,b\in\mathbb{Z}^\times$ such that $\frac{a}{b}=r \Rightarrow \frac{a^2}{b^2}=r^2=p$. However, we know from the previous proof that this latter equation can never be true. Therefore $\sqrt{p}\notin\mathbb{Q}$.
        \end{proof}
    \end{answer}
%\end{problemgroup}

\subsection{Modular Arithmetic}
\subsubsection{Congruence}

%\begin{problemgroup}
    \setcounter{problem}{28}
    \begin{problem}
        Which of the following statements are true?
        \begin{multicols}{3}  
            \begin{enumerate}[label=(\alph*)]
                \item $6 \stackrel{?}{\equiv} 42 \pmod{2}$
                \item $6 \stackrel{?}{\equiv} 43 \pmod{2}$
                \item $7 \stackrel{?}{\equiv} 108 \pmod{10}$
                \item $7 \stackrel{?}{\equiv} 117 \pmod{10}$
                \item $2 \stackrel{?}{\equiv} 54 \pmod{3}$
                \item $2 \stackrel{?}{\equiv} 56 \pmod{3}$
            \end{enumerate}
        \end{multicols}
    \end{problem}
    \begin{answer}
        Statement (a) is true as $2\mid (42-6) \Rightarrow 2\mid 36$ thus both are congruent to $[0] \pmod{2}$. Statement (b) is false as $[6]\equiv[0]\not\equiv[1]\equiv[43]\pmod{2}$. Statement (c) is also false as $[7]\not\equiv [8]\equiv [108] \pmod{10}$. Statement (d) however is true as $10 \mid (117-7) \Rightarrow 10 \mid 110$ thus both are congruent to $[7]\pmod {10}$. Statement (e) is false as $[2]\not\equiv [0]\equiv[54] \pmod{3}$. Finally, Statement (f) is true as $3\mid(56-2)\Rightarrow 3\mid 54$ thus both are congruent to $[2]\pmod{3}$. \par To summarize:
        \begin{multicols}{3}
            \begin{enumerate}[label=(\alph*)]
                \item $6 \equiv 42 \pmod{2}$
                \item $6 \not\equiv 43 \pmod{2}$
                \item $7 \not\equiv 108 \pmod{10}$
                \item $7 \equiv 117 \pmod{10}$
                \item $2 \not\equiv 54 \pmod{3}$
                \item $2 \equiv 56 \pmod{3}$
            \end{enumerate}
        \end{multicols}
    \end{answer}
%\end{problemgroup}
\vspace{5pt}
%\begin{problemgroup}
    \setcounter{problem}{31}
    \begin{problem}
        Complete the following:
        \begin{enumerate}[label=(\alph*)]
            \item If $x \equiv 2 \pmod{5}$, what is $3x^4+x^3+2x-6$ congruent to modulo 5?
            \item If $x \equiv 3 \pmod{6}$, what is $2x^{47891}+5x^3+2x+1$ congruent to modulo 6?
        \end{enumerate}
    \end{problem}
    \begin{answer}
        Please refer to the answers of Problem 43, as they are the same problem phrased differently.
    \end{answer}
%\end{problemgroup}
\vspace{5pt}
%\begin{problemgroup}
    \setcounter{problem}{33}
    \begin{problem}
        Find an example of the following:
        \begin{enumerate}[label=(\alph*)]
            \item $ab \equiv 0 \pmod{n}$ but $a,b \not\equiv 0$.
            \item $ab \equiv ac \pmod{n}$ with $a \not\equiv 0$  and $b \not\equiv c$.
            \item $a^2 \equiv b^2 \pmod{n}$ but $a \not\equiv\pm b$. Hint: Look in a nonprime modulus.
        \end{enumerate}
    \end{problem}
    \begin{answer}
        For part (a) we need to find \( a, b, n \) such that \( ab \equiv 0 \pmod{n} \) but neither \( a \) nor \( b \) are congruent to 0 modulo \( n \). A valid example is: $$
            n = 6, \ a = 2, \ b = 3 \text{ then } 2 \times 3 = 6 \equiv 0 \pmod{6},
        $$
        but neither \( 2 \equiv 0 \pmod{6} \) nor \( 3 \equiv 0 \pmod{6} \). For part (b) we need to find \( a, b, c, n \) such that \( ab \equiv ac \pmod{n} \) with \( a \not\equiv 0 \) and \( b \not\equiv c \). A valid example is: $$
            n = 10, \ a = 2, \ b = 3, \ c = 8 \text{ then } 2 \times 3 = 6 \equiv 16 = 2 \times 8 \pmod{10}, 
        $$ $$
            \text{ then } b = 3 \not\equiv 8 = c \pmod{10}.
        $$  
        Here, \( a = 2 \) is not invertible modulo 10 since \( \gcd(2,10) = 2 \). For part (c) we need to find \( a, b, n \) such that \( a^2 \equiv b^2 \pmod{n} \) but \( a \not\equiv \pm b \). A valid example is: $$
            n = 15, \ a = 4, \ b = 11 \text{ then } 4^2 = 16 \equiv 121 = 11^2 \pmod{15},
        $$ but \( 4 \not\equiv 11 \pmod{15} \) and \( 4 \not\equiv -11 \equiv 4 \pmod{15} \).
    \end{answer}
    
%\end{problemgroup}
\vspace{5pt}
%\begin{problemgroup}
    \setcounter{problem}{35}
    \begin{problem}
        Recall that our decimal system is a base 10 system. For example, this means that 5672 is the decimal representation of $5 \cdot 10^3 + 6 \cdot 10^2 + 7 \cdot 10^1 + 2 \cdot 10^0$. The numbers 5, 6, 7, 2 are called the \emph{digits} of the number. In general, let $n\in \mathbb{Z}_{\geq 0}$ and write its decimal representation as $a_N \dots a_2a_1a_0$ so that $n = \sum_{k=0}^{N}a_k 10^k$. Use modular arithmetic to verify the following rules.
        \begin{enumerate}[label=(\alph*)]
            \item For $k\in\mathbb{N}$, show $k \mid n \Leftrightarrow n\equiv 0 \pmod{k}$.
            \item Show $2 \mid n \Leftrightarrow 2$ divides the last digit of $n$, i.e., if and only $a_0$ is even.
            \item Show $3 \mid n \Leftrightarrow 3$ divides the sum of the digits of $n$, i.e., if and only if $3 \mid \sum_{k=0}^{N}a_k$. Hint: Notice $10\equiv1\pmod{3}$.
            \item If $n$ has a decimal representation of $a_4a_3a_2a_1a_0$, show $7 \mid n \Leftrightarrow 7 \mid (-3a_4 - a_3 + 2a_2 + 3a_1 +a_0)$. Describe the general pattern.
        \end{enumerate}
    \end{problem}
    \begin{answer}
        The following four proofs will answer each respective subproblem.
        \begin{proof}
            Suppose $k \mid n$. This implies $ka=n$ for $a\in\mathbb{Z}$. Thus $ka\equiv 0 \pmod{k} \Rightarrow n \equiv 0 \pmod{k}$ as any multiple of $k$ is congruent to 0. Therefore $k \mid n \Leftrightarrow n\equiv 0 \pmod{k}$.
        \end{proof}
        \begin{proof}
            Suppose $2\mid n$. we can then say that $2\mid\sum_{k=0}^{N}a_k 10^k$. For $a_0$, $k=0$ in our summation and thus $2\mid a_0 10^0 \Rightarrow a \mid a_0$. We also know that any further components of the summation are also divisible by 2 as $2 \mid 10$. Therefore $2 \mid n \Leftrightarrow 2$ divides $a_0$ if and only if $a_0$ is even.
        \end{proof}
        \begin{remark}
            It seems that invoking the definition of an even number would sufficiently prove the statement, as of course 2 divides the last digit if the last digit is even.
        \end{remark}
        \begin{proof}
            Suppose $3 \mid \sum_{k=0}^{N}a_k 10^k$. Invoking the result from the first proof yields $\sum_{k=0}^{N}a_k 10^k \equiv 0 \pmod{3}$ which can be simplified to $\sum_{k=0}^{N}a_k \equiv 0 \pmod{3}$ as 10 raised to any power in $\mathbb{N}$ is congruent to 0 modulo 3.
        \end{proof}
        \begin{proof}
            Suppose there exists a number $n$ with decimal representation $a_4a_3a_2a_1a_0$ such that $7\mid n$. This five digit number can be re-expressed in summation form as $a_4(10^4) + a_3(10^3) + a_2(10^2) + a_1(10^1) + a_0(10^0)$. Invoking the result from the first proof, we can now say that $a_4(10^4) + a_3(10^3) + a_2(10^2) + a_1(10^1) + a_0(10^0) \equiv 0 \pmod{7}$. Each of these coefficients can be rewritten modulo 7, yielding the sequence, in ascending order of $k$, as $$
                1, 3, 2, 6, 4, 5, \dots \pmod{7}
            $$ or rather, $$
                1, 3, 2, -1, -3, -2, \dots \pmod{7}
            $$ Thus the sequence above represents the pattern of coefficients of $a_k10^k$ repeating every 6 digits.
        \end{proof}
        \begin{remark}
            I hope all of these proofs are sufficient enough to show the biconditional relationship established in the subproblems. Any more writing would seem redundant.
        \end{remark}
    \end{answer}
%\end{problemgroup}

\subsubsection{Congruence Classes}

%\begin{problemgroup}
    \setcounter{problem}{38}
    \begin{problem}
        Complete the following:
        \begin{enumerate}[label=(\alph*)]
            \item In $\mathbb{Z}_5$, which sets are the same as $[2]$:
            \begin{enumerate}[label=(\roman*)]
                \item $\{12+5k \mid k \in \mathbb{Z}\}$
                \item $\{14+5k \mid k \in \mathbb{Z}\}$
                \item $\{27-5k \mid k \in \mathbb{Z}\}$
            \end{enumerate}
        \end{enumerate}
    \end{problem}
    \begin{answer}
        The first and third sets are congruent to $[2]$ as $5|(12-2) \Rightarrow 5\mid 10$ and $5|(27-2) \Rightarrow 5\mid 25$. The second set is not congruent to $[2]$ as $5 \nmid (14-2) \Rightarrow 5 \nmid 12$.
    \end{answer}
%\end{problemgroup}
\vspace{5pt}
%\begin{problemgroup}
    \setcounter{problem}{40}
    \begin{problem}
        Let $m,n\in\mathbb{N}$. Attempt to define a map $f\operatorname{:}\mathbb{Z}_m \mapsto \mathbb{Z}_n$ by setting $f([a]_m)=[a]_n$ for $[a]_m \in \mathbb{Z}_m$.
        \begin{enumerate}[label=(\alph*)]
            \item Show this supposed function may not be well defined by finding an example in which $[a]_n \neq [a+m]_n$.
            \item Show $f$ is a well defined function if and only if $n \mid m$. Hint: Show $f$ is well defined if and only if, for each $k \in \mathbb{Z}$ (especially $k=1$), $a+km=a+jn$ for some $j\in \mathbb{Z}$.
        \end{enumerate}
    \end{problem}
    \begin{answer}
        \begin{proof}
            Suppose that $f\operatorname{:}\mathbb{Z}_m\mapsto \mathbb{Z}_n$ such that $f([a]_m)=[a]_n$ for $[a]_m\in \mathbb{Z}_m$. Consider the valid case $f([4]_{10})=[4]_{3}\equiv[1]_{3}$. However $[4]_{3}\neq [4 + 10]_{3}\equiv [2]_3$; therefore the map is not a well defined function. The function $f$ is well-defined if for all $a \in \mathbb{Z}$ and any integer $k$, $$
                [a]_m = [a+km]_m \Rightarrow [a]_n = [a+km]_n.
            $$ This condition holds if and only if: $$
                a \equiv a + km \pmod{n} \quad \forall k \in \mathbb{Z}.
            $$ Simplifying, we get: $$
                km \equiv 0 \pmod{n}.
            $$ This must be true for all integers $k$ particularly for $k = 1$, which means: $$
                m \equiv 0 \pmod{n} \quad \Rightarrow \quad n \mid m.
            $$ Conversely, if $n \mid m$, then $m = qn$ for some $q \in \mathbb{Z}$, so: $$
                a + km \equiv a + kqn \equiv a \pmod{n}.
            $$
            Thus, $f([a]_m) = [a]_n$ is well-defined. Therefore, $f$ is a well-defined function if and only if $n \mid m$.
        \end{proof}
    \end{answer}
%\end{problemgroup}

\subsubsection{Arithmetic}

%\begin{problemgroup}
    \begin{problem}
        Write out the entire addition and multiplication tables for:
        \begin{enumerate}[label=(\alph*)]
            \item $\mathbb{Z}_4$,
            \item \dots
        \end{enumerate}
    \end{problem}
    \begin{answer}
        % Z4
        The following are the addition and multiplication tables ($\mathbb{Z}_4, +)$ and $(\mathbb{Z}_4, \cdot)$, respectively. \\
        \begin{center}
            \begin{tabular}{c|cccc}
                $+$ & 0 & 1 & 2 & 3 \\
                \hline
                0 & 0 & 1 & 2 & 3 \\
                1 & 1 & 2 & 3 & 0 \\
                2 & 2 & 3 & 0 & 1 \\ 
                3 & 3 & 0 & 1 & 2
            \end{tabular}
            \hspace{5pt}
            \begin{tabular}{c|cccc}
                $\cdot$ & 0 & 1 & 2 & 3 \\
                \hline
                0 & 0 & 0 & 0 & 0 \\
                1 & 0 & 1 & 2 & 3 \\
                2 & 0 & 2 & 0 & 2 \\ 
                3 & 0 & 3 & 2 & 1
            \end{tabular}
        \end{center}
    \end{answer}
%\end{problemgroup}
\vspace{5pt}
%\begin{problemgroup}
    \begin{problem}
        Complete the following:
        \begin{enumerate}[label=(\alph*)]
            \item If $x=[2]\in \mathbb{Z}_5$, simplify $[3]x^4+x^3+[2]x-[6]$.
            \item If $x=[3]\in\mathbb{Z}_6$, simplify $[2]x^{47,891} + [5]x^3 + [2]x + [1]$.
        \end{enumerate}
    \end{problem}
    \begin{answer}
        For subproblem (a), we can calculate the expression as
        \begin{align*}
            [3]x^4+x^3+[2]x-[6] &= [3][2]^4 + [2]^3 + [2][2]-[6] \\
            &= [3\cdot 2^4] + [2^3] + [2\cdot 2] - [6] \\
            &= [48 + 8 + 4 - 6] \\
            &= \boxed{[54] = [4] \pmod{5}}.
        \end{align*}
        The expression from subproblem (b) can be calculated similarly, but it's important to note that $[3]^n=[3]\ \forall n\in\mathbb{Z}_{>0}$. Thus we calculate the expression as
        \begin{align*}
            [2]x^{47,891}+[5]x^3+[2]x+[1] &= [2][3]^{47,891}+[5][3]^3 +[2][3]+[1] \\
            &= [2 \cdot 3] + [5 \cdot 3] + [2 \cdot 3] + [1] \\
            &= [6 + 15 + 6 + 1] \\
            &= \boxed{[28] = [4] \pmod{6}}
        \end{align*}
    \end{answer}
%\end{problemgroup}
\vspace{5pt}
%\begin{problemgroup}
    \begin{problem}
        Since $\mathbb{Z}_n$ has only $n$ elements, it is possible to solve an explicit equation simply by substituting in all possible values of $\mathbb{Z}_n$ and checking for success. Use this method to find all solutions to the following equations.
        \begin{enumerate}[label=(\alph*)]
            \item $x^3 + x^2 + x = [0]$ in $\mathbb{Z}_4$.
            \item \dots
        \end{enumerate}
    \end{problem}
    \begin{answer}
        There are only four possible values for $x$:
        \begin{itemize}
            \item $x=0$: $[0]^3+[0]^2 + [0] = [0]$
            \item $x=1$: $[1]^3+[1]^2 + [1] = [3] \not\equiv [0]$
            \item $x=2$: $[2]^3+[2]^2 + [2] = [8 + 4 + 2] = [14] \equiv [2] \not\equiv [0]$
            \item $x=3$: $[3]^3+[3]^2 + [3] = [27 + 9 + 3] = [39] \equiv [3] \not\equiv [0]$
        \end{itemize}
        Therefore $[0]^3+[0]^2 + [0] = [0] \Rightarrow x = [0]$.
    \end{answer}
%\end{problemgroup}

\iffalse

\lhead{Rawson Duplantis}
\chead{MTH 4329: Complex Variables}
\rhead{\today}
%\begin{problemgroup}
    \begin{problem}
        Prove that if $z_1$, $z_2\in\mathbb{C}$ and $z_1z_2=0$ then either $z_1=0$ or $z_2=0$.
    \end{problem}
    \begin{proof}
        Suppose that $\exists\ z_1$, $z_2\in\mathbb{C} \land z_1z_2=0$. We can express their product as
        $$(x_1, y_1)(x_2, y_2) = (x_1x_2 - y_1y_2, x_2y_1 + y_2x_1)= (0,0)$$ which implies $x_1x_2 - y_1y_2 = x_2y_1+y_2x_1$. From this, rearranging yields
        \begin{align*}
            x_1x_2 - y_1y_2 &= x_2y_1 + y_2x_1 \\
            x_1x_2 - y_2x_1 &= x_2y_1 + y_1y_2 \\
            x_1(x_2 - y_2) &= y_1(x_2 + y_2).
        \end{align*}
        Since there is no choice of $x_2,y_2\in\mathbb{Z}\backslash\{0\}$ that makes $x_2 - y_2=x_2 + y_2$ true, $(x_1, y_1)=0 \lor (x_2, y_2)=0$. Thus there are no proper zero divisors in $\mathbb{C}$.
    \end{proof}
%\end{problemgroup}
\begin{remark}
    Is it sufficient to say that $\mathbb{C}$ is a field, all fields do not have proper zero divisors, thus $\mathbb{C}$ has no proper zero divisors?
\end{remark}
\vspace{.5em}
%\begin{problemgroup}
    \begin{problem}
        Do each of the following:
        \begin{enumerate}[label=(\alph*)]
            \item Write the complex number $\frac{1+2i}{3+4i}$ in the form $a+bi$.
            \item Find $\operatorname{Re}\left(\frac{2-i}{2+i}\right)$ and $\operatorname{Im}\left(\frac{2-i}{2+i}\right)$.
        \end{enumerate}
    \end{problem}
    For the first subproblem, we will re-express the fraction using the denominators conjugate, $3-4i$:
    \begin{align*}
        \left(\frac{1+2i}{3+4i}\right)\left(\frac{3-4i}{3-4i}\right) &= \frac{(1+2i)(3-4i)}{9-16i^2} \\
        &= \frac{3+2i-8i^2}{25} \\
        &= \frac{11}{25}+\frac{2}{25}i.
    \end{align*}
    \framebox[1.1\width]{Therefore $a=\frac{11}{25}$ and $b=\frac{2}{25}$.} For the second subproblem, we will complete the same computation but will present the real and imaginary components separately. Using the denominator's conjugate of $2-i$: $$ \left(\frac{2-i}{2+i}\right)\left(\frac{2-i}{2-i}\right) = \frac{4-4i+i^2}{4 - i^2} = \frac{3-4i}{5}. $$ \framebox[1.1\width]{Therefore $\operatorname{Re}\left(\frac{2-i}{2+i}\right)=\frac{3}{5}$ and $\operatorname{Im}\left(\frac{2-i}{2+i}\right)=\frac{-4}{5}$.}
%\end{problemgroup}
\vspace{.5em}
%\begin{problemgroup}
    \begin{problem}
        Prove that if $|z|=2$, then $$\frac{1}{|z^4-4z^2+3|}\leq\frac{1}{3}.$$ (Hint: factor $z^4-4z^2+3$ and then use the reverse triangle inequality.)
    \end{problem}
    \begin{proof}
        Suppose $\exists z\in\mathbb{C}$ where $|z|=2$. Given a polynomial $z^4-4z^2+3$, we can determine using the reverse triangle inequality that
        \begin{align*}
            |z^4-4z^2+3| &\geq |z^4|-|-4z^2|-|3| \\
                         &\geq |2^4-4(2)^2-3| \\
                         &\geq 3
        \end{align*}
        Therefore $|z^4-4z^2+3|^{-1}\leq\frac{1}{3}$.
    \end{proof}
    This result can be confirmed using the modulus-polynomial method---if that accurately refers to the procedure---used in the textbook by finding the moduli of the factors of $|z^4-4x^2+3|$, $|z^2-3|$ and $|z^2-1|$, as 1 and 3 respectively. We then take the greatest number of the two as our reciprocal polynomial's upper bound, in this case $\frac{1}{3}$.
%\end{problemgroup}

\vspace{1em}
\begin{problem}
    Prove the following:
    \begin{enumerate}[label=(\alph*)]
        \item $z$ is real if and only if z=$\bar{z}$.
        \item $z$ is either real or purely imaginary if and only if $(\bar{z})^2=z^2$.
    \end{enumerate}
\end{problem}
\begin{proof}
    Suppose $\exists z$ such that $z \in \mathbb{R}$. Such a real number in the complex plane doesn't have an imaginary component and is represented simply as $(x,0)$; because $\operatorname{Im}(z)$ is the opposite of its conjugate, a real number remains a real number as 0 is neutral. If we are to suppose that $z=\bar{z}$, $\operatorname{Im}(z)$ and $\operatorname{Im}(\bar{z})$ are implied to be the same, which is only true for the neutral zero as $\operatorname{Im}(z)=-\operatorname{Im}(\bar{z})$. Therefore $z\in\mathbb{R}\iff z=\bar{z}$.
\end{proof}
\begin{proof}
    Suppose $\exists z \in \mathbb{C}$. If we also suppose that such a number represented in its component form $a+bi$ is either completely real or completely imaginary, either $z=a$ or $z=bi$ with corresponding $\bar{z}=a$ or $\bar{z}=-bi$. In the real case, $z^2=a^2=(\bar{z})^2$ while in the 'imaginary' case $z^2=(bi)^2=(-bi)^2=(\bar{z})^2$. As clearly demonstrated by the commutativity of the second expression, the reverse relationship is true and thus $z$ is either real or purely imaginary $\iff (\bar{z})^2=z^2$.
\end{proof}

\vspace{1em}
\begin{problem}
    With $z=x+yi$, write the equation $|2\bar{z}+i|=4$ in terms of $x$ and $y$. Sketch the graph of this equation and identify what kind of equation it is.
\end{problem}
If we are to substitute $z=x+yi$ into $|2\bar{z}+i|=4$, we can say $$|2(x-yi)+i|=4\implies |2x-2yi+i|=4.$$ We can calculate the modulus of this expression as the following: $$|2x + (1-2y)i|=4\implies \sqrt{4x^2 + 1 - 4y + 4y^2}=4\implies x^2+y^2-y-\frac{15}{4}=0.$$ Further simplification involves completing the square by separating $\frac{1}{4}$ out: $$x^2+(y^2-y+\frac{1}{4})-\frac{16}{4}=0\implies x^2+(y-\frac{1}{2})^2=4.$$ \framebox[1.1\width]{Therefore $|2\bar{z}+1|=4$ is a circle of radius 2 centered around $(0,\frac{1}{2})$.}

% \maketitle
\section{Intro to LaTeX}
Hi there! Here is a brief introduction to LaTeX. It's a great system for creating beautifully typeset scientific documents. The main difference between LaTeX and your standard word processing program, is that you write code that tells LaTeX what you want it to display and the program creates handles all the little things that make your math look great.

Lets start with some basics. You can type text normally but when you want to type some math start and end your expression with a \$. For example if I type the Pythagorean Theorem, as long as I surround my equation with dollar signs I get $a^2 +b^2 = c^2$. If I want to emphasize an equation simply put TWO dollar signs around the mathematics. For example a similar expression with two dollar signs around it looks like so: $$x_1^n + x_2^n = x_3^n.$$ (check out the code to see how to generate subscripts!)

In LaTeX we use the curly braces, \{ \}, to group things. If we want to raise $e$ to the $42$nd power and we just type it with no grouping we get $e^42$ which is not what we want. Surrounding the $42$ with curly braces gives $e^{42}$.

The really really really great thing about latex is the HUGE library of mathematical symbols, every symbol starts with a $\backslash$, so if I want a nice pi, I just type $\backslash$pi in math mode like so: $\pi$. Want an integral? Try $\backslash$int, want a fraction? Try $\backslash$frac\{a\}\{b\} (this gives $\frac{a}{b}$) Here is a more complicated example:
$$
\int_a^b f \left(\frac{x}{2} \right) \ dx =2\left( F\left(\frac{b}{2} \right) - F\left(\frac{a}{2}\right) \right)
$$
(those nice sized parenthesis come from using the $\backslash$left( and $\backslash$right) commands). If you are wondering what the latex command for a symbol is, just try and guess it. If that doesn't work, google it!

There are a few little tricks to making your math look nice, one is the align environment which lets you type multiple lines of aligned mathematical expressions. For instance:
\begin{align*}
(x^2+y^2)(z^2+w^2) & = (xz)^2 + (yz)^2 + (xw)^2 + (yw)^2 \\
& =  (xz)^2 + 2wxyz + (yw)^2 + (yz)^2 - 2 wxyz + (xw)^2 \\
&= (xz+yw)^2 + (yz-xw)^2
\end{align*}
here the $\backslash\backslash$ is a line break and the $\&$ is an alignment character. 
\section{Sample Proofs}
\begin{problem} $1^3+2^3+\ldots +n^3 = \left[ \frac{n(n+1)}{2}\right]^2$  for all natural numbers. 

\end{problem}
 
\begin{proof}
We proceed by induction. When $n=1$ we have 
$$
1^3 + \ldots + n^3 = 1^3 =1
$$
and
$$
\left[ \frac{n(n+1)}{2}\right]^2=\left[ \frac{1(1+1)}{2}\right]^2 = 1^2 =1.
$$
Thus the equation holds when $n=1$.

Now assume the equation holds for some $n \in \mathbb N$. Then by our assumption we have
\begin{align*}
1^3 + \ldots + n^3 + (n+1)^3& =   \left[ \frac{n(n+1)}{2}\right]^2 + (n+1)^3 = (n+1)^2\left( \frac{n^2}{4} + (n+1)\right) \\
&=(n+1)^2\left( \frac{n^2+4n+4}{4} \right) = (n+1)^2\frac{(n+2)^2}{4}  \\ &=  \left[ \frac{(n+1)(n+2)}{2}\right]^2.
\end{align*}
Therefore, by induction, the equation holds for all $n\in \mathbb N$.
\end{proof}

\begin{problem} Prove $A \subseteq B \Rightarrow A \cap C \cont B \cap C$
\end{problem}
\begin{proof} Assume $A \cont B $ and let $x\in A \cap C$. If $x \in  A \cap C$, then $x\in A$ and $x \in C$. As $x\in A$, by assumption we have that $x \in B$. Thus $x \in B$ and $x \in C$, giving $x\in B \cap C$. Therefore if $A \subseteq B $, then $A \cap C \cont B \cap C$.
\end{proof}

% THE DOCUMENT IS ESSENTIALLY DONE AT THIS POINT, NO NEED TO EDIT ANYTHING BELOW THIS______________________________________________________________________________________________
 
\fi
\end{document}